\documentclass[a4wide]{article}
\usepackage{a4wide}
\usepackage{appendix}
\newcommand{\comment}[1]{{\tt #1}}
\usepackage{graphicx}
\usepackage{listings}
\usepackage[section]{placeins}
\title{RocketMouse:\\ A Game for Assessing ADHD}

\author{Hesham H. Salman \and Jonathon Kissinger \and Sean Mead \and Troy Johnson}

\begin{document}
\maketitle


\section{Abstract}
\- Studies show attention deficit disorders affects upwards of 5\% of the population. In recent years, mobile games have also become extremely popular due to the increasing ubiquity of smart phones. By exploiting this trend of mobile game popularity, we hope to create new tools to measure the symptoms of ADHD in the form of a game. In this paper, we explore the results of testing response time, distractibility, and app preference in users with and without ADHD through their reactions times and other traits collected through a mobile game.  We predict that users with ADHD type disorders will have higher variability in reaction times, higher distractibility, and will prefer mobile applications that do not focus on reaction time as a measure of success. Previous studies have indicated that the inhibitory abilities of children with ADHD are normalized when playing video games. Due to this, our mobile game will alternate between levels that are more game-like and more test-like. We also predict that children diagnosed with ADHD will have more apps installed on their smartphones as their short attention spans may cause them to jump from app to app.

\section{Introduction}
\- With the recent explosion in the ubiquity of mobile devices and the mobile games that often accompany those mobile devices, a great opportunity is presented to researchers for reaching a large scale, geographically widespread audience through mobile applications, especially for those limited by their local geographic location. Studies show attention deficit disorders affects upwards of 5\% of the population and typically most users have to make a trip to the doctor's office to have their symptoms measured and have a diagnosis made.
\newline
\newline
\- We propose the development of a mobile application that can help collect data about users to help quantify the response time characteristics that may often accompany a person with ADHD. The mobile application would first survey the user about things such as their age, gender, ADHD diagnosis, if they're medicated for ADHD, and other basic questions. Once the user is done filling out the survey, the user is allowed the play the game. The game will record the response time characteristics of the user as they are playing and upload the response time data to a remote server them once the user is done playing. From there, data mining is performed on the user’s response time data to see if any patterns can be established in combination with user's data to see if any strong patterns in the response times stick out between users with ADHD and users without ADHD. The game should be fun and engaging to keep the user interested and playing and hopefully collect as much data on their response time characteristics as possible.
\newline
\newline
\- Our game implementation will be a two-dimensional side scrolling game in which the user controls a character that moves from left to right across the screen. As the user moves from left to right across the screen, they must jump over obstacles that appear very quickly, this is where the recording of response times comes into play. The time it takes for the user to tap the screen to make their character jump from the time the object first appears is the time that we will use as a measure for user's reaction time. Once the user is done, their response time data will be uploaded to a data for further analysis. We predict that users with ADHD type disorders will have higher variability in these reaction times, higher distractibility, and will prefer mobile applications that do not focus on reaction time as a measure of success.

\section{Background Studies}
\- There have been many studies done that revolve around reaction time and ADHD. However, we have been unable to find any mobile applications that have been developed for the sole purpose of collecting data to help with ADHD diagnosis. Many studies have indicated that here is in fact a link between the reaction time of people with and without ADHD.
\newline
\newline
\- In [1] the authors conducted a study on 151. Out of the 151 children, 104 were diagnosed with ADHD, the other 47 were diagnosed as not having ADHD. The children performed a variety of tasks, but the task that displayed the biggest difference between the two sets of children was the Go/No Go task. The Go/No-Go test is a visual reaction test. The p value was < 0.01 with the null hypotheses being that there was no difference between the two groups of children. So this test was pretty strong evidence that there existed a difference between reaction times for the ADHD diagnosed children and non ADHD diagnosed children.
\newline
\newline
\- In [2] the authors showed that there appears to exist a relationship between the variability in reaction times and children diagnosed with ADHD. The study consisted of 144 participants, 60 of which weren’t diagnosed with ADHD, and 84 of which had previously been diagnosed with ADHD. The participants all participated in four reaction time tasks in which they had incentive to do well because they earned prizes if they did well. Overall, the study concluded that ADHD was associated with slower reaction time and also a higher variability in reaction times.
\newline
\newline
\- CogCubed is a game that was developed for attempting to diagnose people with ADHD [3]. However, CogCubed required a patient to go into a doctor's office to use, so the user couldn't have access to the tool outside of a doctor's office. This is where our research contribution could be really strong. Having a mobile application that is fun could lead to the contribution of a lot of data that could be analyzed and mined, which could hopefully lead to stronger conclusions or maybe even new conclusions on the relationship between reaction times and ADHD diagnosis.

\section{Methods}
\subsection{Planning of Game and Development Environment}
Unity? iOS/Swift?
\subsection{Database Design}
\- We will have one database and that database will be housed on the server. The database will consist of three tables; reaction time table, a survey table, and a user table. It is important to note that we don't store any identifying data about the user such as their name of anything. We assign each user a unique user id for the purpose of organizing the data and associating each user with their reaction time data, survey data, and in the event that they submit more data, we can just add the new data to their data already on the server instead of creating a new entry in the user table. The reaction time table will house the data on the reaction times the user had in the game. The survey table will contain the answers to the survey, and the user table will store data on the user such as their unique user id.
\subsection{Privacy and Security of User's Data}
\- Due to the sensitive nature of the data collected, our mobile application must be secure and also must not store any data locally on the mobile device, except for the circumstance where there are multiple users on the same device. In this case, an anonymous identifier will be generated locally to differentiate between the different users. 
\subsection{Server set up}
\- A server had to be set up to accept data from the user's mobile device, which is uploading the response time data from the user for further processing. We will be utilizing a LAMP set up for the server.
\subsection{Development of game}
\- The gather data from user, we thought that the best way to get a large pool of data would be to develop a mobile application that is a game. This way, it would be entertaining for the user to play and the user may be willing to submit more data by playing longer or more often. To develop the game, we first needed to decide on the game type, we decided on a 2d side scroller where the response time characteristics are determined by the amount of time it takes the user the press the screen to jump over an object once the object has appeared on screen. 
\subsection{Server side scripts for analyzing data}
\- Server side scripts are the scripts that will reside on the server which are responsible for processing and analyzing the response time data that the users send to the server from their mobile devices. The scripts will be responsible for accepting data, organizing the data, and storing the data into the SQL database located on the server. The scripts will also be responsible for pulling data from the server and mining the returned data for patterns.  


\section{References}
[1] Epstein JN, Langberg JM, Rosen PJ, et al. Evidence for higher reaction time variability for children with ADHD on a range of cognitive tasks including reward and event rate manipulations. Neuropsychology. 2011;25(4):427-41.\newline
[2] Andreou P, Neale BM, Chen W, et al. Reaction time performance in ADHD: improvement under fast-incentive condition and familial effects. Psychol Med. 2007;37(12):1703-15.\newline
[3] Monika D. Heller, Kurt Roots, Sanjana Srivastava, Jennifer Schumann, Jaideep Srivastava, and T. Sigi Hale. Games for Health Journal. October 2013, 2(5): 291-298. doi:10.1089/g4h.2013.0058.\newline

\end{document}
